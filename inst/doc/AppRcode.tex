\section{\ttt{R}--Code} \label{app:R_code}
The following is all the \ttt{R}--code used in the tutorial:

\refNumberName{sec:preliminaries}
\begin{verbatim}
##A qqplot for some N(0,1) random numbers
qqnorm(rnorm(100))

##Summary of 100 N(0,1) numbers.
summary(rnorm(100))

##Load the spatio-temporal package
library(SpatioTemporal)

##A package used for some plots in this tutorial (plotCI)
library(plotrix) 
###The maps to provide reference maps
library(maps)

##Load the data
data(mesa.data)
##...and the precomputed results
data(mesa.data.res)
\end{verbatim}
\refNumberName{sec:examine_data}
\begin{verbatim}
names(mesa.data)
\end{verbatim}
\refNumberName{sec:mesa_data_frame}
\begin{verbatim}
head(mesa.data$location)
\end{verbatim}
\ttt{\#\#\#Plot the locations, see \autoref{fig:map}}%
\vspace*{-1\baselineskip}
\begin{verbatim}
par(mfrow=c(1,1))
plot(mesa.data$location$long,mesa.data$location$lat,
     pch=24,bg=c("red","blue")[mesa.data$location$type],
     xlab="Longitude",ylab="Latitude")

###Add the map of LA
map("county","california",col="#FFFF0055",fill=TRUE,add=TRUE)

##Add a legend
legend("bottomleft",c("AQS","FIXED"),pch=24,bty="n",
       pt.bg=c("red","blue"))
\end{verbatim}
%$
\refNumberName{sec:LUR_data_frame}
\begin{verbatim}
head(mesa.data$LUR)
\end{verbatim}
%$
\refNumberName{sec:trend_data_frame}
\begin{verbatim}
head(mesa.data$trend)
range(mesa.data$trend$date)
\end{verbatim}
%$
\refNumberName{sec:mesa.data.obs}
\begin{verbatim}
head(mesa.data$obs)
\end{verbatim}
%$
\refNumberName{sec:ST_array}
\begin{verbatim}
dim(mesa.data$SpatioTemp)
mesa.data$SpatioTemp[1:5,1:5,]

str(dimnames(mesa.data$SpatioTemp))
as.character(sort(unique(c(mesa.data$obs$date,
    mesa.data$trend$date))))
dimnames(mesa.data$SpatioTemp)[[3]]
\end{verbatim}
%$
\refNumberName{sec:data_summary}
\begin{verbatim}
printMesaDataNbrObs(mesa.data)
\end{verbatim}
\ttt{\#\#\#Plot when observations occurr, see \autoref{fig:time_space}}%
\vspace*{-1\baselineskip}
\begin{verbatim}
par(mfcol=c(1,1),mar=c(4.3,4.3,1,1))
plotMonitoringLoc(mesa.data)
\end{verbatim}
\refNumberName{sec:raw_data}
\begin{verbatim}
data(mesa.data.raw)
names(mesa.data.raw)
names(mesa.data)  ###Recall the data frames in mesa.data

head(mesa.data.raw$X)

mesa.data.raw$obs[1:6,1:5]

mesa.data.raw$lax.conc.1500[1:6,1:5]
\end{verbatim}
%$
\refNumberName{sec:creating_location}
\begin{verbatim}
##extract components from mesa.data.raw
##to create the location data.frame
location <- mesa.data.raw$X[,c("ID", "x", "y", "long",
                               "lat", "type")]
##ensure that it's a data.frame
location <- as.data.frame(location)
##ensure that ID and type are factors
location$ID <- factor(as.character(location$ID))
location$type <- factor(as.character(location$type))
\end{verbatim}
%$
\refNumberName{sec:creating_LUR}
\begin{verbatim}
##create a covariate LUR matrix
LUR <- mesa.data.raw$X
##ensure that it's a data.frame
LUR <- as.data.frame(LUR)
##take rownames as the IDs and drop location information
rownames(LUR) <- as.character(LUR$ID)
LUR <- LUR[,!(names(LUR) %in% c("ID", "x", "y", "long",
           "lat", "type"))]
\end{verbatim}
\refNumberName{sec:creating_obs}
\begin{verbatim}
##create a data.frame of observations
T <- t(matrix(rownames(mesa.data.raw$obs),
              nrow=dim(mesa.data.raw$obs)[2],
              ncol=dim(mesa.data.raw$obs)[1],byrow=TRUE))
ID <- matrix(colnames(mesa.data.raw$obs), 
             nrow=dim(mesa.data.raw$obs)[1],
             ncol=dim(mesa.data.raw$obs)[2],byrow=TRUE)
obs <- data.frame(obs=c(mesa.data.raw$obs), date=c(as.Date(T)), 
                  ID=c(ID))
##drop unmonitored locations
obs <- obs[!is.na(obs$obs),,drop=FALSE]
##sort the locations (strictly not needed)
obs <- obs[order(obs$date,obs$ID),,drop=FALSE]

##examine the time and location matrices
T[1:3,1:3]
ID[1:3,1:3]
\end{verbatim}
\refNumberName{sec:creating_ST}
\begin{verbatim}
##create a 3D-array for the spatio-temporal covariate
ST <- array(mesa.data.raw$lax.conc.1500, dim =
            c(dim(mesa.data.raw$lax.conc.1500),1))
##add names for time, location and spatio-temporal covariate
dimnames(ST) <- list(rownames(mesa.data.raw$lax.conc),
                     colnames(mesa.data.raw$lax.conc),
                     "lax.conc.1500")
\end{verbatim}
\refNumberName{sec:creating_trend}
\begin{verbatim}
##compute the smooth trends
trend <- calc.smooth.trends(obs=obs$obs, date=obs$date,
                            ID=obs$ID, n.basis = 2)$svd

##Combining all the elements
mesa.data.alt <- list(location=location, LUR=LUR, trend=trend,
                      obs=obs, SpatioTemp=ST)
\end{verbatim}
\refNumberName{sec:data_set_up}
\begin{verbatim}
##extract a data matrix
D <- create.data.matrix(mesa.data)
##and study the data
dim(D)
D[1:6,1:5]
colnames(D)

##subset the data
ID.subset <-mesa.data$location[mesa.data$location$type ==
                               "AQS",]$ID
D2 <- create.data.matrix(mesa.data, subset=ID.subset)
##and study the result
dim(D2)
colnames(D2)
\end{verbatim}
\refNumberName{sec:CV_trend_est}
\begin{verbatim}
##Run leave one out cross-validation to find smooth trends
SVD.cv <- SVD.smooth.cv(D,1:5)
\end{verbatim}
\refNumberName{sec:evaluating_CV_smooth}
\vspace*{\baselineskip}
\\ \ttt{\#\#Plotting the smooth trends CV results, see \autoref{fig:svd_cv}}%
\vspace*{-1\baselineskip}
\begin{verbatim}
par(mfcol=c(2,2),mar=c(4,4,.5,.5))
plot(SVD.cv$CV.stat$RMSE, type="l", ylab="RMSE", xlab="")
plot(SVD.cv$CV.stat$R2, type="l", ylab="R2", xlab="")
plot(SVD.cv$CV.stat$BIC, type="l", ylab="BIC", xlab="")
\end{verbatim}
\ttt{\#\#Scatterplots of BIC values for different number of trends,}\\
\ttt{\#\#see \autoref{fig:BIC_pairs}}%
\vspace*{-1\baselineskip}
\begin{verbatim}
par(mfcol=c(1,1),mar=c(4,4,.5,.5))
pairs(SVD.cv$BIC.all, pane = 
      function(x,y){points(x,y);abline(0,1)})
\end{verbatim}
%$
\refNumberName{sec:creating_and_investigating_trends}
\begin{verbatim}
##compute new temporal smooths
F <- calc.smooth.trends(mesa.data, n.basis=2)
##and add the new trends to the data structure
mesa.data$trend <- F$svd
\end{verbatim}
\ttt{\#\#plot the observations at two of the locations along} \\
\ttt{\#\#with the fitted smooth trends, see \autoref{fig:smooth_trends}}%
\vspace*{-1\baselineskip}
\begin{verbatim}
par(mfrow=c(4,1),mar=c(2.5,2.5,2,.5))
plotMesaData(mesa.data, 5, type="obs")
plotMesaData(mesa.data, 18, type="obs")

##also plot the residuals
plotMesaData(mesa.data, 5, type="res")
plotMesaData(mesa.data, 18, type="res")
\end{verbatim}
\ttt{\#\#Autocorrelation of the residuals at four locations,}\\
\ttt{\#\#see \autoref{fig:autocorr}}%
\vspace*{-1\baselineskip}
\begin{verbatim}
par(mfcol=c(2,2),mar=c(2.5,2.5,3,.5))
plotMesaData(mesa.data, 1, type="acf")
plotMesaData(mesa.data, 5, type="acf")
plotMesaData(mesa.data, 13, type="acf")
plotMesaData(mesa.data, 18, type="acf")
\end{verbatim}
\refNumberName{sec:beta_fields}
\begin{verbatim}
##create data matrix
D <- create.data.matrix(mesa.data)
beta <- matrix(NA,dim(D)[2], dim(mesa.data$trend)[2])
beta.std <- beta
##extact the temporal trends
F <- mesa.data$trend
##drop the date column
F$date <- NULL

##estimate the beta-coeficients at each location
for(i in 1:dim(D)[2]){
  tmp <- summary(lm(D[,i] ~ as.matrix(F)))
  beta[i,] <- tmp$coefficients[,1]
  beta.std[i,] <- tmp$coefficients[,2]
}

##Add names to the estimated betas
colnames(beta) <- c("const",colnames(F))
rownames(beta) <- colnames(D)
dimnames(beta.std) <- dimnames(beta)

##examine the beta fields
head(beta)
head(beta.std)
\end{verbatim}
\refNumberName{sec:par_estimation}
\begin{verbatim}
##Examine available spatial, ...
names(mesa.data$LUR)
##... and spatio-temporal covariates
dimnames(mesa.data$SpatioTemp)[[3]]

##specify the model
mesa.data.model <- create.data.model(mesa.data, LUR = 
    list(c("log10.m.to.a1", "s2000.pop.div.10000", 
           "km.to.coast"), "km.to.coast", "km.to.coast"), 
    ST.Ind=NULL)

##examine the model
names(mesa.data.model)
mesa.data.model$LUR.list
mesa.data.model$ST.Ind

##covariates for the temporal intercept
head(mesa.data.model$X$const)
##...and the two smooth temporal trends
head(mesa.data.model$X$V1)
head(mesa.data.model$X$V2)

##Some important dimensions of the model
dim <- loglike.dim(mesa.data.model)
print(dim)

##Setting up the initial parameter values for optimization
x.init <- cbind(rep(2,dim$nparam.cov), 
                c(rep(c(1,-3),dim$m+1),-3))

##Names of the covariance parameters
loglike.var.names(mesa.data.model,all=FALSE)

##Add names to the initial values
rownames(x.init) <- loglike.var.names(mesa.data.model,
                                      all=FALSE)
x.init

##estimate parameters
if(FALSE){
  ##This may take a while...
  par.est <- fit.mesa.model(x.init, mesa.data.model, type="p",
      hessian.all=TRUE, control=list(trace=3,maxit=1000))
}else{
  ##Get the precomputed optimisation results instead.
  par.est <- mesa.data.res$par.est
}

##compare function values
loglike(par.est$res.best$par, mesa.data.model)
par.est$res.best$value
\end{verbatim}
%$
\refNumberName{sec:evaluating_results}
\begin{verbatim}
##Optimisation status message
par.est$message

##the optimisation results
names(par.est)

##examine optimisation results
names(par.est$res.best)
names(par.est$res.all[[1]])
names(par.est$res.all[[2]])

##lets compare the estimated parameters
cbind(par.est$res.all[[1]]$par.all,
      par.est$res.all[[2]]$par.all)
##and values of the likelihood
cbind(par.est$res.all[[1]]$value,
      par.est$res.all[[2]]$value)

##extract the estimated parameters
x <- par.est$res.best$par.all

##and approximate uncertainties from the hessian
x.sd <- sqrt(diag(-solve(par.est$res.best$hessian.all)))
names(x.sd) <- names(x)
\end{verbatim}
\ttt{\#\#Plot the estimated parameters with uncertainties, see \autoref{fig:conf_int}}%
\vspace*{-1\baselineskip}
\begin{verbatim}
par(mfrow=c(1,1),mar=c(13.5,2.5,.5,.5))
plotCI(x,uiw=1.96*x.sd,ylab="",xlab="",xaxt="n")
points(par.est$res.all[[2]]$par.all, col=2, pch=4)
abline(h=0, col="grey")
axis(1,1:length(x),names(x),las=2)
\end{verbatim}
%$
\refNumberName{sec:predictions}
\begin{verbatim}
##compute conditional expectations (takes roughly 1 min)
EX <- cond.expectation(par.est$res.best$par, mesa.data.model,
                       compute.beta = TRUE)

##study the results
names(EX)
\end{verbatim}
\ttt{\#\#Plot that compares two estimated beta-fields, see \autoref{fig:betafields}}%
\vspace*{-1\baselineskip}
\begin{verbatim}
par(mfcol=c(1,1), mar=c(4.5,4.5,2,.5), pty="s")
plotCI(x=beta[,1], y=EX$EX.beta[,1],
       uiw=1.96*beta.std[,1], err="x",
       main="Intercept", pch=NA, sfrac=0.005,
       xlab="Empirical estimate",
       ylab="Spatio-Temporal Model")
plotCI(x=beta[,1],y=EX$EX.beta[,1],
       uiw=1.96*sqrt(EX$VX.beta[,1]),
       add=TRUE, pch=NA, sfrac=0.005)
abline(0,1,col="grey")
\end{verbatim}
\ttt{\#\#...and the other two beta-fields, see \autoref{fig:betafields2}}%
\vspace*{-1\baselineskip}
\begin{verbatim}
par(mfcol=c(1,2), mar=c(4.5,4.5,2,.5), pty="s")
for(i in 2:3){
  plotCI(x=beta[,i], y=EX$EX.beta[,i],
         uiw=1.96*beta.std[,i], err="x",
         main=colnames(beta)[i], pch=NA, sfrac=0.005,
         xlab="Empirical estimate",
         ylab="Spatio-Temporal Model")
  plotCI(x=beta[,i],y=EX$EX.beta[,i],
         uiw=1.96*sqrt(EX$VX.beta[,i]),
         add=TRUE, pch=NA, sfrac=0.005)
  abline(0,1,col="grey")
}
\end{verbatim}
\ttt{\#\#plot predictions and observations for 4 locations, see \autoref{fig:pred_TS}}%
\vspace*{-1\baselineskip}
\begin{verbatim}
par(mfrow=c(4,1),mar=c(2.5,2.5,2,.5))
plotPrediction(EX,  1, mesa.data.model)
lines(as.Date(rownames(EX$EX.mu)),EX$EX.mu[,1],col=3)
lines(as.Date(rownames(EX$EX.mu.beta)),
      EX$EX.mu.beta[,1],col=4)
plotPrediction(EX, 10, mesa.data.model)
lines(as.Date(rownames(EX$EX.mu)),EX$EX.mu[,10],col=3)
lines(as.Date(rownames(EX$EX.mu.beta)),
      EX$EX.mu.beta[,10],col=4)
plotPrediction(EX, 17, mesa.data.model)
lines(as.Date(rownames(EX$EX.mu)),EX$EX.mu[,17],col=3)
lines(as.Date(rownames(EX$EX.mu.beta)),
      EX$EX.mu.beta[,17],col=4)
plotPrediction(EX, 22, mesa.data.model)
lines(as.Date(rownames(EX$EX.mu)),EX$EX.mu[,22],col=3)
lines(as.Date(rownames(EX$EX.mu.beta)),
      EX$EX.mu.beta[,22],col=4)
\end{verbatim}
\refNumberName{sec:CV}
\begin{verbatim}
##create the CV structure defining 10 different CV-groups
Ind.cv <- createCV(mesa.data.model, groups=10, min.dist=.1)
head(Ind.cv)

##number of observations in each CV-group
colSums(Ind.cv)

##Which sites belong to which groups?
ID.cv <- lapply(apply(Ind.cv,2,list),function(x) 
                unique(mesa.data.model$obs$ID[x[[1]]]))
ID.cv

##Distances between the sites in the 10th CV-group
##Note that the sites with distance 0.084<min.dist 
##are grouped together.
mesa.data.model$dist[ID.cv[[10]],ID.cv[[10]]]

##Find out which location belongs to which cv group
I.col <- apply(sapply(ID.cv,
    function(x) mesa.data.model$location$ID %in% x),1,
    function(x) if(sum(x)==1) which(x) else 0)
names(I.col) <- mesa.data.model$location$ID
I.col
\end{verbatim}
\ttt{\#\#Plot the locations, colour coded by CV-grouping, see \autoref{fig:mapCV}}%
\vspace*{-1\baselineskip}
\begin{verbatim}
par(mfrow=c(1,1))
plot(mesa.data$location$long,mesa.data$location$lat,
     pch=23+floor(I.col/max(I.col)+.5), bg=I.col, 
     xlab="Longitude",ylab="Latitude")

##Add the map of LA
map("county","california",col="#FFFF0055",fill=TRUE,add=TRUE)
\end{verbatim}
%$
\refNumberName{sec:CV_estimations}
\begin{verbatim}
if(FALSE){
  ##estimate different parameters for each CV-group
  dim <- loglike.dim(mesa.data.model)
  x.init <- cbind(rep(2,dim$nparam.cov),
                  c(rep(c(1,-3),dim$m+1),-3))
  ##This may take a while...
  par.est.cv <- estimateCV(x.init, mesa.data.model, Ind.cv)
}else{
  ##Get the precomputed results instead.
  par.est.cv <- mesa.data.res$par.est.cv
}

##lets examine the results
names(par.est.cv)
par.est.cv$par

##res.all contains a list with the optimisation results 
##for each CV-group
str(par.est.cv$res.all[[1]])

##Studying the conv-field of all the groups
sapply(par.est.cv$res.all,function(x) x$conv)
##we see that all optimisations have converged.

##We can also see that different starting values
##where best for different cv-groups 
sapply(par.est.cv$res.all,function(x) x$par.init)
\end{verbatim}
\ttt{\#\#Boxplot of the different estimates from the CV, see \autoref{fig:CVparest}}%
\vspace*{-1\baselineskip}
\begin{verbatim}
par(mfrow=c(1,1), mar=c(7,2.5,2,.5), las=2)
boxplot(t(par.est.cv$par))
points(par.est$res.best$par, pch=4, col=2)
\end{verbatim}
%$
\refNumberName{sec:CV_predictions}
\begin{verbatim}
##Do cross-validated predictions using the newly 
##estimated paremeters (takes roughly 1 minute)
pred.cv <- predictCV(par.est.cv$par, mesa.data.model,
                     Ind.cv, silent = FALSE)

##examine the results
names(pred.cv)
head(pred.cv$pred.obs)
str(pred.cv$pred.all[[1]])

##compute CV-statistics (see also predictNaive)
##this uses compute.ltaCV internally to compute the
##long term averages if we want just lta:s then use 
##compute.ltaCV directly.
pred.cv.stats <- summaryStatsCV(pred.cv, lta=TRUE, trans=0)
names(pred.cv.stats)
pred.cv.stats$Stats
\end{verbatim}
\ttt{\#\#Plot observations with CV-predictions and}\\
\ttt{\#\#95\% prediction intervals, see \autoref{fig:pred_cv_ci}}%
\vspace*{-1\baselineskip}
\begin{verbatim}
par(mfcol=c(4,1),mar=c(2.5,2.5,2,.5))
plotCV(pred.cv,  1, mesa.data.model)
plotCV(pred.cv, 10, mesa.data.model)
plotCV(pred.cv, 17, mesa.data.model)
plotCV(pred.cv, 22, mesa.data.model)
\end{verbatim}
\ttt{\#\#Plot predicted vs. observed, see \autoref{fig:pred_obs}}%
\vspace*{-1\baselineskip}
\begin{verbatim}
par(mfcol=c(1,1),mar=c(4.5,4.5,2,2))
plot(pred.cv$pred.obs[,"obs"], pred.cv$pred.obs[,"pred"],
     xlab="Observations", ylab="Predictions", pch=19,
     cex=.1, col=mesa.data.model$obs$idx)
abline(0,1,col="grey")
\end{verbatim}
\ttt{\#\#Predicted long term average against observations} \\
\ttt{\#\#(on the natural scale), see \autoref{fig:pred_lta}}%
\vspace*{-1\baselineskip}
\begin{verbatim}
par(mfcol=c(1,1),mar=c(4.5,4.5,3.5,2))
plot(pred.cv.stats$lta[,"obs"], pred.cv.stats$lta[,"pred"],
     xlab="Observations", ylab="Predictions", main="Averages",
     ylim=c(25,95), xlim=c(25,95))
abline(0,1,col="grey")
\end{verbatim}
\refNumberName{sec:CV_residuals}
\begin{verbatim}
##create a vector dividing data into four seasons
I.season <- matrix(NA,length(mesa.data.model$obs$date),1)
I.season[months(mesa.data.model$obs$date) %in%
         c("December","January","February"),1] <- "DJF"
I.season[months(mesa.data.model$obs$date) %in%
         c("March","April","May"),1] <- "MAM"
I.season[months(mesa.data.model$obs$date) %in%
         c("June","July","August"),1] <- "JJA"
I.season[months(mesa.data.model$obs$date) %in%
         c("September","October","November"),1] <- "SON"
I.season <- factor(I.season,levels=c("DJF","MAM","JJA","SON"))

##No. observations in each season
table(I.season)
\end{verbatim}
\ttt{\#\#Residual QQ-plots, see \autoref{fig:qqplots}}%
\vspace*{-1\baselineskip}
\begin{verbatim}
par(mfrow=c(1,2),mar=c(3,2,1,1),pty="s")
##Raw Residual QQ-plot
CVresiduals.qqnorm(pred.cv.stats$res, I.season=I.season)
##Normalized Residual QQ-plot
CVresiduals.qqnorm(pred.cv.stats$res.norm, norm=TRUE,
                   I.season=I.season)
\end{verbatim}
\ttt{\#\#Residual Scatter Plots, see \autoref{fig:resid_scatter}}%
\vspace*{-1\baselineskip}
\begin{verbatim}
par(mfcol=c(2,1),mar=c(4.5,4.5,2,2))
CVresiduals.scatter(pred.cv.stats$res, mesa.data.model$F[,2],
     I.season=I.season, xlab="First temporal smooth",
     main="CV Residuals - All data")
CVresiduals.scatter(pred.cv.stats$res, 
     mesa.data.model$X[[1]][ mesa.data.model$obs$idx, 2],
     I.season=I.season, main="CV Residuals - All data",
     xlab=colnames(mesa.data.model$X[[1]])[2])
\end{verbatim}
\ttt{\#\#Residual Scatter Plots by type of site, see \autoref{fig:resid_scatter_by_type}}%
\vspace*{-1\baselineskip}
\begin{verbatim}
par(mfcol=c(2,2),mar=c(4.5,4.5,2,2))
CVresiduals.scatter(pred.cv.stats$res, 
  mesa.data.model$X[[1]][ mesa.data.model$obs$idx, 2], I.type=
  mesa.data.model$location$type[mesa.data.model$obs$idx],
  I.season=I.season, xlab=colnames(mesa.data.model$X[[1]])[2], 
  main="CV Residuals - ")
\end{verbatim}
