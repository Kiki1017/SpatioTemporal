\section{Simulation} \label{app:simulation}
Another option for evaluating model behaviour is to use simulated data.
Instead of using actual observations and comparing predictions to actual 
observations, we simulate \lno{x} observations using the same model as
in \autoref{sec:estimation}, and comparing predictions to the simulated data.

\subsection{Load Data}
Let us first load relevant libraries and data.
\vspace*{-0.5\baselineskip}
\begin{verbatim}
##Load the spatio-temporal package
library(SpatioTemporal)
##And additional packages for plotting
library(plotrix) 
library(maps)

##load the data model (same as before)
data(mesa.data)
data(mesa.data.model)
##...and optimisation results
data(mesa.data.res)
\end{verbatim}

\subsection{Simulating some Data}
First we simulate 4 samples of new data, using the parameters previously 
estimated in \autoref{sec:par_estimation}.
\vspace*{-0.5\baselineskip}
\begin{verbatim}
##Extract parameters
x <- mesa.data.res$par.est$res.best$par

##And simulate new data using previously estimated parameters
sim.data <- simulateMesaData(x,  mesa.data.model, rep=4)

##examine the result
names(sim.data)
str(sim.data,1)

##Here sim.data$X contains the 4 simulations, sim.data$B 
##contains the simulated beta fields and sim.data$obs 
##contains observations data.frames that can be used to 
##replace mesa.data.model$obs.

##lets create model structures that contain the simulated data
mesa.data.sim <- list()
for(i in 1:length(sim.data$obs)){
  ##copy the mesa.data.model object
  mesa.data.sim[[i]] <- mesa.data.model
  ##replace observations with the simulated data
  mesa.data.sim[[i]]$obs <- sim.data$obs[[i]]
}

##Compute predictions for the 4 simulated datasets.
##Here we'll just use the known parameters, however one could 
##easily estimate new parameters based on the simulated data 
##using fit.mesa.model (although this would take more time)

##Please note that following the predictions take roughly 
##3 minutes on a decent laptop.
E <- list()
for(i in 1:length(sim.data$obs)){
  E[[i]] <- cond.expectation(x, mesa.data.sim[[i]], 
                             compute.beta = TRUE)
}
\end{verbatim}
%$

\subsection{Studying the Results}
Given simulated datasets and predictions based on the simulated data
we study how well the estimates agree with the simulated data.
\vspace*{-0.5\baselineskip}
\begin{verbatim}
##First we compare the simulated, known values of the first 
##beta-field with the predictions.
par(mfrow=c(2,2),mar=c(4.5,4.5,2,.5))
for(i in 1:4){
  ##plot confidence intervalls for the predicted data
  plotCI(E[[i]]$EX.beta[,"const"], 
         uiw = 1.96*sqrt(E[[i]]$VX.beta[,"const"]),
         ylab="Beta-field", xlab="Locations",
         main=sprintf("Simulation set %d",i))
  ##and add the known values from the simulation.
  points(sim.data$B[,"const",i],col="red",pch=19,cex=.5)
}

##Let's compare the predicted values and the simulated data for 
##all four simulations at three different sites
##First site
par(mfrow=c(2,2),mar=c(2.5,2.5,2,.5))
for(i in 1:4){
  ##plot predictions, but not the observations
  plotPrediction(E[[i]], "60590001", mesa.data.sim[[i]], 
                 lty=c(1,NA))
  ##add the simulated data (i.e. observations + 
  ##simulated values at points where we've predicted)
  lines(as.Date(rownames(sim.data$X)), 
        sim.data$X[,"60590001",i], col="red")
}

##Second site
par(mfrow=c(2,2),mar=c(2.5,2.5,2,.5))
for(i in 1:4){
  plotPrediction(E[[i]], "60371602", mesa.data.sim[[i]], 
                 lty=c(1,NA))
  lines(as.Date(rownames(sim.data$X)), 
        sim.data$X[,"60371602",i], col="red")
}

##Third site
par(mfrow=c(2,2),mar=c(2.5,2.5,2,.5))
for(i in 1:4){
  plotPrediction(E[[i]], "L002", mesa.data.sim[[i]],
                 lty=c(1,NA))
  lines(as.Date(rownames(sim.data$X)), 
        sim.data$X[,"L002",i], col="red")
}

##Finally we also compare the predicted long term average
##at each site with the average of the simulated data.
par(mfrow=c(2,2),mar=c(4.5,4.5,2,.5))
for(i in 1:4){
  plot(apply(sim.data$X[,,i],2,mean),apply(E[[i]]$EX,2,mean),
       xlab="Simulated data", ylab="Predictions", 
       main = sprintf("Long term average for simulation %d",i))
  abline(0,1,col="grey")
}
\end{verbatim}
%$

\subsection{Simulation at Unobserved Locations}
Then we setup the data structures --- dropping some observations to simulate 
a case with observed and unobserved locations --- and study the 
available data.
\vspace*{-0.5\baselineskip}
\begin{verbatim}
##store the original data structure
mesa.data.org <- mesa.data

##keep only observations from the AQS sites
##This gives us 5 "unobserved" sites at which to predict
ID.AQS <- mesa.data$location$ID[mesa.data$location$type=="AQS"]
mesa.data$obs <- mesa.data$obs[mesa.data$obs$ID %in% ID.AQS,]

##study the reduced data structure, we see 
##that the 5 FIXED sites lack observations.
printMesaDataNbrObs(mesa.data)
##compate this to the original data.
printMesaDataNbrObs(mesa.data.org)

##create an object containing only the unmonitored sites
mesa.unmon <- mesa.data
ID.miss <- !(mesa.unmon$location$ID %in% 
             unique(mesa.unmon$obs$ID))
mesa.unmon$location <- mesa.unmon$location[ID.miss,]
mesa.unmon$LUR <- mesa.unmon$LUR[mesa.unmon$location$ID,]
mesa.unmon$SpatioTemp <- 
    mesa.unmon$SpatioTemp[,mesa.unmon$location$ID,,drop=FALSE]
##drop observations from the unmonitored data
mesa.unmon$obs <- NULL

##study the data structure for the unobserved locations
printMesaDataNbrObs(mesa.unmon)

##create a model object, dropping unobserved locations.
##Dropping the unobserved locations will speed up the parameter 
##estimations and is thus often a good idea.
mesa.data.model <- create.data.model(mesa.data,
  LUR = mesa.data.model$LUR.list, 
  ST.Ind = mesa.data.model$ST.Ind)

##note that this drops unobserved sites from the data structure
printMesaDataNbrObs(mesa.data.model)

##And now simulate new data using previously 
##estimated parameters
sim.data.obs <- simulateMesaData(x, mesa.data.model)
sim.data.unobs <- simulateMesaData(x, mesa.data.model, 
                                   mesa.data=mesa.unmon)
sim.data.all <- simulateMesaData(x, mesa.data.model, 
      mesa.data=mesa.unmon, combine.data=TRUE)

##This results in simulations at
##All sites
colnames(sim.data.all$X)
##... only observed sites
colnames(sim.data.obs$X)
##... and only unobserved sites
colnames(sim.data.unobs$X)

##note that the unobserved sites do not have a $obs vector
sim.data.unobs$obs
\end{verbatim}