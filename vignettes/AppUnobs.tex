\section{Prediction at Unobserved Locations} \label{app:pred_unobs}
The following is an example of predictions at unobserved locations and times.

\subsection{Load Data}
Let us first load relevant libraries and data.
\vspace*{-0.5\baselineskip}
\begin{verbatim}
##Load the spatio-temporal package
library(SpatioTemporal)
##And additional packages for plotting
library(plotrix) 

##load data
data(mesa.data)
##...and optimisation results
data(mesa.data.res)
\end{verbatim}

\subsection{Setup and Study the Data}
Then we setup the data structures --- dropping some observations to simulate 
a case with observed and unobserved locations --- and study the 
available data.
\vspace*{-0.5\baselineskip}
\begin{verbatim}
##store the original data structure
mesa.data.org <- mesa.data

##keep only observations from the AQS sites
##This gives us 5 "unobserved" sites at which to predict
ID.AQS <- mesa.data$location$ID[mesa.data$location$type=="AQS"]
mesa.data$obs <- mesa.data$obs[mesa.data$obs$ID %in% ID.AQS,]

##Let us also expand the temporal trends to be every week
##instead of every 2-weeks, giving us some unobserved 
##time-points at which to predict.
T <- seq(min(mesa.data$trend$date), max(mesa.data$trend$date),
         by=7)
mesa.data$trend <- data.frame(
    V1=spline(x=mesa.data$trend$date, y=mesa.data$trend$V1, 
              xout=T)$y, 
    V2=spline(x=mesa.data$trend$date, y=mesa.data$trend$V2, 
              xout=T)$y, date=T)

##study the reduced data structure, we see 
##that the 5 FIXED sites lack observations.
printMesaDataNbrObs(mesa.data)
##compate this to the original data.
printMesaDataNbrObs(mesa.data.org)

##create an object containing only the unmonitored sites
mesa.unmon <- mesa.data
ID.miss <- !(mesa.unmon$location$ID %in% 
             unique(mesa.unmon$obs$ID))
mesa.unmon$location <- mesa.unmon$location[ID.miss,]
mesa.unmon$LUR <- mesa.unmon$LUR[mesa.unmon$location$ID,]
mesa.unmon$SpatioTemp <- 
    mesa.unmon$SpatioTemp[,mesa.unmon$location$ID,,drop=FALSE]
##drop observations from the unmonitored data
mesa.unmon$obs <- NULL

##study the data structure for the unobserved locations
printMesaDataNbrObs(mesa.unmon)

##create a model object, dropping unobserved locations.
##Dropping the unobserved locations will speed up the parameter 
##estimations and is thus often a good idea.
mesa.data.model <- create.data.model(mesa.data,
    LUR = list(c("log10.m.to.a1", "km.to.coast", 
      "s2000.pop.div.10000"), "km.to.coast","km.to.coast"),
    ST.Ind = NULL)

##create a model object containing all the sites (strip=FALSE)
mesa.data.model.all <- create.data.model(mesa.data,
    LUR = list(c("log10.m.to.a1", "km.to.coast", 
      "s2000.pop.div.10000"), "km.to.coast","km.to.coast"),
    ST.Ind = NULL, strip=FALSE)

##note that this drops unobserved sites from the data structure
printMesaDataNbrObs(mesa.data.model)
printMesaDataNbrObs(mesa.data.model.all)
\end{verbatim}

\subsection{Predictions}
Having created data structures that contain some unobserved locations 
and time-points we are now ready to compute predictions at these unobserved 
points.

In a real world application we would probably use \ttt{fit.mesa.model} 
to estimate parameters. However, here we will just use the parameters 
previously estimated in \autoref{sec:par_estimation} using all available 
data.
\vspace*{-0.5\baselineskip}
\begin{verbatim}
##Strictly we should now estimate the relevant parameters.
##However since this is time consuming, we instead use the
##parameters estimated for the model using all observations.
x <- mesa.data.res$par.est$res.best$par

##Now let's use these parameters to compute som predictions
##Please note that the predictions take roughly 30s each on a 
##decent laptop.

##predict at times and locations in mesa.data.model (AQS-sites)
E.AQS <- cond.expectation(x, mesa.data.model, 
                          compute.beta = TRUE)

##predict at times and locations in mesa.unmon (FIXED-sites)
E.FIXED <- cond.expectation(x, mesa.data.model, 
    mesa.data = mesa.unmon, compute.beta = TRUE)

##predict at times in mesa.data.model and locations in both 
##mesa.data.model and mesa.unmon (AQS + FIXED-sites)
##using the "combine.data" option.
E.ALL <- cond.expectation(x, mesa.data.model, 
    mesa.data = mesa.unmon, compute.beta = TRUE, 
    combine.data = TRUE)

##predict at times and locations in mesa.data.model.all (the 
##one with strip=FALSE), this gives predictions att all 
##locations and *times*.
E.ALL2 <- cond.expectation(x, mesa.data.model.all, 
                           compute.beta = TRUE)

##To predict at unmonitored sites for only 2000, we first 
##pick out the relevant time points in mesa.unmon$trend
mesa.unmon$trend <- mesa.data$trend
mesa.unmon$trend <- 
    mesa.unmon$trend[mesa.unmon$trend$date>="2000-01-01" &
                     mesa.unmon$trend$date<"2001-01-01",]
##and then predict at these locations
E.2000 <- cond.expectation(x, mesa.data.model, 
    mesa.data = mesa.unmon, compute.beta = TRUE)
\end{verbatim}

\subsection{Results}
Having computed predictions in a number of different ways we 
now want to investigate the results.
\subsubsection{Studying the Results}
We start by looking at the predictions produced for the five different cases.
\vspace*{-0.5\baselineskip}
\begin{verbatim}
##First look at the components of the prediction structure
names(E.AQS)

##Here E.AQS$EX is a matrix containing predictions at all 
##relevant times and locations. The dimension of the matrix 
dim(E.AQS$EX)
colnames(E.AQS$EX)
rownames(E.AQS$EX)[1:5]
##And we see that E.AQS has predicted at the 20 locations 
##and 280 time-points in mesa.data.model.

##While E.FIXED contains predictions at the 5 locations 
##and 559 time-points (weekly data) in mesa.unmon
dim(E.FIXED$EX)

##Using the "combine.data" option gives predictions at 
##the 20+5=25 locations in mesa.data.model and mesa.unmon 
##BUT this uses the temporal trends from mesa.data.model 
##resulting in predictions at ONLY 280 time-points
dim(E.ALL$EX)

##If we instead use mesa.data.model.all which contains all 
##the 25 locations as well as all the 559 time-points we 
##obtain predictions at ALL locations and time-points.
dim(E.ALL2$EX)

##Finally we also see that restricting mesa.unmon$trend to 
##only 2000 dates gives the expected 5 locations and 52 
##(weekly trend in mesa.unmon) timepoints
dim(E.2000$EX)
\end{verbatim}
%$

\subsubsection{Plotting the Results}
Having studied the number of locations and time-points at which we 
have predictions we now investigate the actual predictions. 

Recall that we dropped all observations for the {\sc fixed}-sites to simulate 
a case where predictions are desired at some locations given observations 
at other locations. However, we still have the observations at the 
{\sc fixed}-sites in the \ttt{mesa.data.org} structure, this allows us to add 
the known but {\emph unused} observations to the following plots, 
providing a measure of how good the predictions are.
\vspace*{-0.5\baselineskip}
\begin{verbatim}
##Start by plotting predictions and observations 
##for 3 observed locations
par(mfrow=c(3,1),mar=c(2.5,2.5,2,.5))
plotPrediction(E.AQS, "60371103", mesa.data.model)
plotPrediction(E.AQS, "60371601", mesa.data.model)
plotPrediction(E.AQS, "60590001", mesa.data.model)

##Using the E.ALL strucutre that has predictions at 
##all locations and on the 2-week time scale gives 
##an identical result.
par(mfrow=c(3,1),mar=c(2.5,2.5,2,.5))
plotPrediction(E.ALL, "60371103", mesa.data.model.all)
plotPrediction(E.ALL, "60371601", mesa.data.model.all)
plotPrediction(E.ALL, "60590001", mesa.data.model.all)

##Before plotting predictions for the unobserved 
##locations we do a few minor things

##studying the distance between the observed and 
##unobserved locations
apply(create.data.model(mesa.data.org)$dist[1:20,21:25],2,min)
##we note that L001 is very close (0.083 km) to an AQS site.

create.data.model(mesa.data.org)$dist[,"L001"]
##Which on closer inspection turns out to be site 60371103.

##Let's also extract the data matrix from the original 
##data so that we can compare the predicted values with 
##the known, LEFT OUT observations
D <- create.data.matrix(mesa.data.org)

##We are now ready to plot predictions for three unobserved 
##locations along with the left out obserbations (and for L001 
##the data from the colocated site). Plotting the left out 
##observations allows us to investigate how well the
##predictions are doing

par(mfrow=c(3,1),mar=c(2.5,2.5,2,.5))
#site L001
plotPrediction(E.FIXED, "L001", mesa.unmon)
##add the colocated data
lines(as.Date(rownames(D)), D[,"60371103"], col="green")
##also add the left out observations
lines(as.Date(rownames(D)), D[,"L001"], col="red")

##and the other two sites
plotPrediction(E.FIXED, "LC002", mesa.unmon)
lines(as.Date(rownames(D)), D[,"LC002"], col="red")
plotPrediction(E.FIXED, "LC003", mesa.unmon)
lines(as.Date(rownames(D)), D[,"LC003"], col="red")

##Now plotting with the prediction structure that has computed 
##predictions on the weekly time-scale. Since we now have 
##observed only every other point standard usage of 
##plotPreditcion will not show any observations (a line with 
##only every other point...)
par(mfrow=c(4,1),mar=c(2.5,2.5,2,.5))
plotPrediction(E.ALL2, "60371103", mesa.data.model.all)
plotPrediction(E.ALL2, "60590001", mesa.data.model.all)
##Instead we need to plot the observations as points.
plotPrediction(E.ALL2, "60371103", mesa.data.model.all,
               lty=c(1,NA), pch=c(NA,19), cex=c(NA,.5))
plotPrediction(E.ALL2, "60590001", mesa.data.model.all,
               lty=c(1,NA), pch=c(NA,19), cex=c(NA,.5))
\end{verbatim}
