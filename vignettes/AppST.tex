\section{Modelling with a Spatio-Temporal covariate}  
\label{app:pred_ST_covar}
The following is an example of modelling with a spatio\hyp{}temporal covariate.

\subsection{Load Data}
Let us first load relevant libraries and data.
\vspace*{-0.5\baselineskip}
\begin{verbatim}
##Load the spatio-temporal package
library(SpatioTemporal)
##And additional packages for plotting
library(plotrix) 

##load data
data(mesa.data)
data(mesa.data.model)
##...and optimisation results
data(mesa.data.res)
\end{verbatim}

\subsection{Setup and Study the Data}
Then we setup the data structures, creating a new \ttt{mesa.data.model} 
that contains the spatio\hyp{}temporal covariate.
\vspace*{-0.5\baselineskip}
\begin{verbatim}
##create model structure with ST-covariate
mesa.data.model.ST <- create.data.model(mesa.data,
    LUR = mesa.data.model$LUR.list, 
    ST.Ind="lax.conc.1500")

##An alternative is to seperate the spatio-temporal 
##covariate into an average over time, and a mean-zero 
##spatio-temporal covariate.
##First we create a new data object
mesa.data.mean0 <- remove.ST.mean(mesa.data)

##with mean-zero spatio-temporal covariate.
colMeans(mesa.data.mean0$SpatioTemp)

##The mean has been added as a geographic covariate
names(mesa.data.mean0$LUR)

##create model structure with mean-zero ST-covariate
mesa.data.model.ST.mean0 <- create.data.model(mesa.data.mean0,
    LUR = list(c("log10.m.to.a1", "s2000.pop.div.10000",
                 "km.to.coast", "mean.lax.conc.1500"),  
               "km.to.coast", "km.to.coast"), 
    ST.Ind="lax.conc.1500")

##note that the models have different number of covariates
dim <- loglike.dim(mesa.data.model)
dim.ST <- loglike.dim(mesa.data.model.ST)
dim.ST0 <- loglike.dim(mesa.data.model.ST.mean0)

##number of spatio-temporal covariates
c(dim$L, dim.ST$L, dim.ST0$L)

##number of geogrpahic covariates for the 
##two temporal trends+intercept
cbind(dim$p, dim.ST$p, dim.ST0$p)

##The models require the same number of log-covariance 
##parameters (same number of beta-fields)
c(dim$nparam.cov, dim.ST$nparam.cov, dim.ST0$nparam.cov)

##... but different number of regression parameters
c(dim$nparam, dim.ST$nparam, dim.ST0$nparam)
\end{verbatim}
%$

\subsection{Parameter Estimation}
Given the two models that contain spatio\hyp{}temporal covariates we estimate
parameters for the models (or load precomputed results).
\vspace*{-0.5\baselineskip}
\begin{verbatim}
##create initial values
dim <- loglike.dim(mesa.data.model.ST)
x.init <- c(rep(c(1,-3),dim$m+1),-3)

##estimate parameters
if(FALSE){
  ##This may take a while...
  par.est.ST <- fit.mesa.model(x.init, mesa.data.model.ST,
        hessian.all=TRUE, control=list(trace=3,maxit=1000))

  par.est.ST.mean0 <- fit.mesa.model(x.init,
        mesa.data.model.ST.mean0, hessian.all=TRUE,
        control=list(trace=3,maxit=1000))
}else{
  ##Get the precomputed optimisation results instead.
  par.est.ST <- mesa.data.res$par.est.ST
  par.est.ST.mean0 <- mesa.data.res$par.est.ST.mean0
}
##and the reference model
par.est <- mesa.data.res$par.est
\end{verbatim}
%$

\subsection{Predictions}
Having created two models and estimated parameters we are now ready 
to compute some predictions for the models.
\vspace*{-0.5\baselineskip}
\begin{verbatim}
##extract the estimated parameters
x.ST <- par.est.ST$res.best$par.all
x.ST0 <- par.est.ST.mean0$res.best$par.all

##as well as parameters for the model without a 
##spatio-temporal covariate
x <- par.est$res.best$par.all

##compute some predictions for these three models
EX <- cond.expectation(x, mesa.data.model, 
                       compute.beta = TRUE)
EX.ST <- cond.expectation(x.ST, mesa.data.model.ST,
                          compute.beta = TRUE)
EX.ST.0 <- cond.expectation(x.ST0, mesa.data.model.ST.mean0,
                            compute.beta = TRUE)
\end{verbatim}
%$

\subsection{Results}
Having estimated the models and computed predictions we 
now want to investigate the results.

\subsubsection{Estimation Results }
We start by looking at estimated parameters for the two different cases 
with spatio\hyp{}temporal covariates, and compare to the model without
spatio\hyp{}temporal covariates.
\vspace*{-0.5\baselineskip}
\begin{verbatim}
##first that both optimisations have converged
par.est.ST$message
par.est.ST.mean0$message

##extract the estimated parameters
x.ST <- par.est.ST$res.best$par.all
x.ST0 <- par.est.ST.mean0$res.best$par.all
x <- par.est$res.best$par.all

##and parameter uncertainties
x.ST.sd <- sqrt(diag(-solve(par.est.ST$res.best$hessian.all)))
x.ST0.sd <- 
  sqrt(diag(-solve(par.est.ST.mean0$res.best$hessian.all)))
x.sd <- sqrt(diag(-solve(par.est$res.best$hessian.all)))

##plot the estimated parameters
par(mfrow=c(1,1),mar=c(13.5,2.5,.5,.5))
plotCI(1:19, x.ST0, uiw=1.96*x.ST0.sd, 
       ylab="", xlab="", xaxt="n")
plotCI(c(1:5,7:19)-.2, x.ST, uiw=1.96*x.ST.sd, 
       add=TRUE, col=2)
plotCI(c(2:5,7:19)+.2, x, uiw=1.96*x.sd, 
       add=TRUE, col=3)
abline(h=0, col="grey")
axis(1,1:length(x.ST0),names(x.ST0),las=2)
legend("bottomleft", col = c(1,2,3), pch = 1, 
       legend=c("par.est.ST","par.est.ST.mean0","par.est.ST"))
\end{verbatim}
In this case the including of a spatio\hyp{}temporal covariate hardly 
affects the parameter estimates at all. It should be noted that
this spatio\hyp{}temporal covariate has been included in the data
{\em to provide an example} of how a covariate can be used by 
the model.

\subsubsection{Prediction Results}
Having studied the estimated parameters we now 
investigate the resulting predictions. 
\vspace*{-0.5\baselineskip}
\begin{verbatim}
##Start by plotting predictions and observations 
##for 1 locations, for the three models
par(mfrow=c(3,1),mar=c(2.5,2.5,2,.5))
plotPrediction(EX, "60590001", mesa.data.model)
plotPrediction(EX.ST, "60590001", mesa.data.model)
plotPrediction(EX.ST.0, "60590001", mesa.data.model)

\end{verbatim}

\vspace*{-0.5\baselineskip}
\begin{verbatim}
##Start by plotting predictions and observations 
##for 3 observed locations
par(mfrow=c(3,1),mar=c(2.5,2.5,2,.5))
plotPrediction(E.AQS, "60371103", mesa.data.model)
plotPrediction(E.AQS, "60371601", mesa.data.model)
plotPrediction(E.AQS, "60590001", mesa.data.model)

##Due to the minimal influence of the spatio-temporal 
##covariate there is little difference between the 
##different predictions, both for the observations
range(EX$EX-EX.ST$EX)
range(EX$EX-EX.ST.0$EX)

##...and for the latent beta-fields
range(EX$EX.beta-EX.ST$EX.beta)
range(EX$EX.beta-EX.ST.0$EX.beta)
\end{verbatim}
